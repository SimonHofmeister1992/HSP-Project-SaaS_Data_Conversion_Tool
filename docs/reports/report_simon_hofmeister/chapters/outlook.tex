\section{Ausblick}
In der derzeitigen Fassung, ist es möglich, Daten über die graphische Benutzeroberfläche von einem Service auszulesen, in die MongoDB zu schreiben, zu filtern, transformieren und in eine weitere API zu schreiben. Hierfür existieren derzeit sechs Services, drei für Kalender und drei für Notizen. In Zukunft ist es vorstellbar, dass hierfür innerhalb der Kategorien viel mehr Services zur Verfügung stehen, und auch Services für neue Kategorien erstellt werden, um weitere Transformationen zu ermöglichen. Zudem bietet auch die GUI einen großen Spielraum, so könnten weitere Filtermöglichkeiten für das Datum oder die Verarbeitung komplexer Attribute ermöglicht werden. Mit einem weiteren Eingabefeld, könnte man es Experten auch gestatten, selbst weitere Stages für die MongoDB Aggregation-Pipeline zu schreiben um viel mächtigere Transformationen zu gewährleisten. Weiter wäre denkbar, einmal getätigte Mappings zwischen Attributen für spätere Aufrufe zu speichern, sodass der Anwender dieses nicht erneut durchführen muss. \\
Wie zu erkennen ist, hat die Anwendung ein sehr großes Potential mit vielen Erweiterungsmöglichkeiten. Sobald genug Services und eine Verbesserung der GUI, im Sinne von einer einfacheren oder mächtigeren Handhabbarkeit, vorhanden sind, ist es denkbar, dass das Tool auch im privaten oder professionellen Bereich zum Zweck von regelmäßigen Datensicherungen oder Datentransformationen genutzt wird. Da es immer mehr verschiedene, spezialisierte Software-as-a-Service-Tools gibt, würde dieses Tool als Schnittstelle einen großen Vorteil bringen, so dass jeder seine Daten für sich wie gewünscht organisieren kann.