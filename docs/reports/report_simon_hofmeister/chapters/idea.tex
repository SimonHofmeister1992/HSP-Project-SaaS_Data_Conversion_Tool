\section{Idee des Projekts}
Auf dem Markt existieren bereits sehr viele Tools, die Daten zwischen SaaS-Services transferieren und es erlauben die Datenströme zu manipulieren. In diesem Kapitel wird beschrieben, warum dieses Projekt zustande kam, welche Problemstellungen gelöst werden und wie das Tool eingesetzt werden kann.

\subsection{Gegebene Problemstellungen}
Auf dem Markt gibt es immer mehr SaaS-Dienste, die wenige spezialisierte Probleme lösen. Durch die hohe Vielzahl an SaaS-Produkten ist es jedoch leicht, den Überblick über alle genutzten Dienste zu verlieren. Somit wäre es für die Benutzer wichtig, die Daten zwischen den Diensten synchronisieren zu können, um den Überblick wiederherzustellen. Ebenso ist es wünschenswert die Daten einfach in einen neuen Dienst migrieren zu können, wenn auf eine neue IT-Lösung umgestellt wird. Bei vielen SaaS-Diensten gibt es jedoch keine Möglichkeit die Daten in einen externen fremden Dienst zu exportieren oder gar zu transformieren, höchstens in die andere Richtung, von einem externen Anbieter zum eigenen Dienst. Um diese Probleme zu lösen, haben sich bereits mehrere Tools, wie ifttt, etabliert. Ifttt observiert einen gegebenen Dienst und erzeugt für jede Änderung, beispielsweise der Erzeugung eines Termins, eine entsprechende Eintragung im Zieldienst. Jedoch befinden sich hierbei die Daten des Benutzers in einem Online-Tool, dass die Services jederzeit observiert, zudem können keine lokalen Dateien wie archivierte Kalenderdateien eingebunden werden.


\subsection{Ziele}
Aus der gegebenen Problemstellung ergibt sich der Wunsch, die Daten der Dienste möglichst einfach in einen beliebigen anderen Service migrieren zu können. Durch die verschiedenen Arbeitsweisen der Nutzer ist es wichtig, diesen ihren bevorzugten Arbeitsstil zu ermöglichen, hierfür müssen nicht nur Daten innerhalb einer Kategorie migriert, sondern auch zwischen mehreren Kategorien transformiert werden, sodass ein Benutzer beispielsweise die Möglichkeit hat, statt mit Terminen mit Notizen zu arbeiten. Die hohe Anzahl an existierenden SaaS-Tools erfordert somit die optimale Nutzung des Potentials eine sehr hohe Flexibilität in der neu geschaffenen Anwendung, um idealerweise jedes beliebige SaaS-Produkt mit aufnehmen zu können. Da das Tool nur von den Schnittstellen der Dienste abhängig ist, und somit meist sowohl lesend als auch schreibend auf die Daten zugreifen kann, bietet sich zudem die einfache Erweiterung, die Daten lokal abzuspeichern um Backups zu ermöglichen.
