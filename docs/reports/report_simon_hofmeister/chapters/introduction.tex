\section{Einführung}
Durch die voranschreitende Digitalisierung, für die ein immer größer werdendes Angebot an verschiedenen Micro-Tools aufgebaut wird, existiert auch eine Vielzahl an spezialisierten online gehosteten Software-as-a-Service-Produkten \cite{Abolhassan.2016}, wie der Exchange-Kalender von Microsoft, bei dem für Meetings nach freien Zeitslots für alle Teilnehmer gesucht werden kann, oder auch der Teamup-Kalender, welcher unter anderem für das Planen von Kursen und die zugehörigen Einschreibungen genutzt werden kann. \\
Durch das große Angebot verteilen sich jedoch auch die Daten der Benutzer, wodurch es schwieriger wird den Überblick zu behalten. Werden mehrere verschiedene Kalender-Tools verwendet, ohne diese korrekt zu synchronisieren, kann es somit schnell der Fall sein, dass ein Termin in einem Zeitslot bestätigt wird, obwohl in diesem bereits ein anderer Termin stattfindet. Andererseits bietet dieses System auch die Möglichkeit, dass jeder Benutzer besser seine bevorzugten Tools nutzen kann, so arbeiten manche Personen gerne mit Kalendern, andere kommen besser mit Notizen zurecht.
Um beiden Problemstellungen gerecht zu werden, ist es notwendig, dass die Anwender selbst bestimmten können, wie sie die Informationen der Tools zusammenfassen oder transformieren möchten. \\
Für diese Problemstellung gibt es mittlerweile zahlreiche Anwendungen, bei denen ein Tool nur als Baustein für ein größeres auf dem Markt. Eines dieser Angebote ist \glqq ifttt \grqq, in langer Schreibweise \glqq if this then that \grqq, bei dem beispielsweise bei jeder Erstellung eines Termins im Google Kalender automatisch eine Notiz in OneNote erzeugt werden kann. \\
Im Gegensatz dazu ist das im nachfolgenden vorgestellte Tool darauf ausgelegt, dass es für eine Datenspeicherung oder Konvertierung direkt vom Benutzer lokal aufgerufen werden muss, und dieser den Datenstrom bei Bedarf manipulieren kann. Durch die lokale Verarbeitung der Daten, geben die Anwender die Kontrolle über die Verarbeitung ihrer Daten nicht an externe Serviceanbieter ab.\\
Im folgenden werden die Organisationsstruktur im Projekt, inklusive der verwendeten Hardware und Software, das Konfigurationsmanagement, ein Überblick über die Projektstruktur, die Kalender-Services von Google, Microsoft Exchange und Teamup, die zugehörige graphische Benutzeroberfläche(GUI) und die verwendete Datenbank MongoDB beschrieben. Die letzteren zwei Kapitel bilden zudem die Schnittstelle zwischen den Kalender- und Notiz-Services ab. Im Anschluss daran befindet sich ein kurzer Ausblick, wie sich das Tool zukünftig entwickeln könnte.