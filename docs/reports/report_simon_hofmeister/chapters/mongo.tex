\section{MongoDB Datenbank}
Die Datenbank MongoDB bildet die zentrale Instanz der Applikation, in die die Datensätze persistiert werden, und von der die Daten wieder abgefragt werden, um diese in einen Service zu schreiben. Hierbei bietet MongoDB den Vorteil, dass die Daten in einem JSON-ähnlichen Format gespeichert, das auch ähnlich zu den python-Dictionaries aufweist und somit ein hohes Maß an Flexibilität für die Anzahl und Art der Attribute mit sich bringt.\\
Zudem bietet MongoDB die Möglichkeit, Daten die abgefragt werden, direkt über eine Aggregation-Pipeline zu transformieren. Dies erleichtert die Transformation für die einzelnen Services, da hierbei direkt Filter, neue Mappings und Aggregationsmethoden angewandt werden können. Hierfür werden drei Arrays für Parameter definiert: Filteroptionen, zusätzliche Pipeline Schritte und Aggregationsoptionen. \\
Die so erhaltenen Datensätze werden nicht komplett in den Speicher geladen, da durch sehr viele gespeicherte Daten-Objekte auch viel Speicherplatz verbraucht werden kann, daher wird der Iterator verwendet um eine gegebene Funktion auf jedes erhaltene python-Dictionary anzuwenden. Diese Funktionalität wird derzeit genutzt, um jeden Datensatz einzeln in einen Service zu schreiben. Als Folge resultiert ein niedrigerer lokaler Speicherverbrauch, jedoch eine deutlich höhere Anzahl an API-Anfragen.  