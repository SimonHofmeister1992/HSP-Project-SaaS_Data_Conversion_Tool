\section{Google Kalender-Service}
Durch die Bekanntheit und Vielzahl der zur Verfügung stehenden Tools, wird der Google-Kalender von einer sehr großen Bevölkerungsgruppe benutzt. Dieser Universalkalender bietet neben dem Erstellen von Terminen auch viele weitere Möglichkeiten, wie die Verknüpfung der Termine mit mehreren Videokonferenzlösungen, beispielsweise Google-Hangouts, dem Hinzufügen von Dateien, optionalen Gästen oder Ressourcen, um nur wenige zu nennen. Mit über 120 trivialen und komplexen Attributen bietet google zudem sehr viele Optionen, wodurch diese Struktur als Mapping-Grundlage für die weiteren Services dient. Um diese Daten abfragen zu können, muss eine credentials.json-Datei von google angefragt werden, die mit dem Projekt verknüpft wird und als API-Key fungiert. Die Autorisierung erfolgt dann beim Aufruf des Services über den Login im Browser, hierbei wird eine Datei namens token.pickle erstellt, mit dessen Hilfe wäre ein weiterer Login bei künftigen Abfragen nicht mehr notwendig, aus Gründen der Datensicherheit und der erwarteten seltenen Nutzungsrate, ist die Speicherung dieser Datei im Quellcode des googleCalendarApiInterface auskommentiert, kann jedoch, wenn gewünscht, wieder genutzt werden.

\subsection{Erstellen einer Authentifizierungs-Datei}
Um Daten vom Google-Kalender abzufragen oder in diesen zu schreiben, muss zuerst die credentials.json-Datei erstellt werden. Hierzu klickt man auf der Website \url{https://developers.google.com/calendar/quickstart/python}, auf den Button \glqq Enable the Google Calendar API-Key \grqq, wählt einen Projektnamen, den Typ der App, in diesem Fall eine Desktop-App und kann nun, nach dem Klick auf \glqq Create \grqq die Datei zur Authentifizierung herunterladen und in den gewünschten Ordner des Projektes, hier src\textbackslash calendar\textbackslash google, verschieben.

\subsection{Lesende und schreibende Abfragen der Google-API}
Um mit dem Google-Kalender zu interagieren, wird aus den Daten der credentials.json-Datei und den Authentifikationsdaten ein service-Objekt erstellt, mit dem verschiedene Abfragen möglich sind. So können beispielsweise alle Termine abgefragt werden, wobei Serientermine entweder nur aus dem Starttermin mit dem Pattern für die Serie oder dekomprimiert mit allen zugehörigen Terminen enthalten sind, oder es werden Termine in den Google-Kalender mithilfe einer insert-Funktionalität zurückgeschrieben. \\
Bei der GET-Abfrage der Daten aus dem Kalender wird eine Liste von python-Dictionary zurückgegeben, das alle vorhandenen Attribute enthält. Die Anfrage benötigt hierbei die Google-ID des abgefragten Kalenders, als Standard wird der Hauptkalender \glqq primary \grqq verwendet. Zudem kann man nicht nach speziellen Events, wie durch die Angabe eines Start-Datums im von Google gebräuchlichen ISO-8601-Formats, suchen, hierfür ist nur die eindeutige ID des Termins zulässig. \\
Um Termine in den Google-Kalender zu schreiben, werden neben dem konkreten Kalender in den geschrieben werden soll, auch die Informationen über das Start- und Enddatum benötigt. Hierfür wird das konkrete Terminobjekt in ein python-dictionary zurückgeschrieben und der Insert-Funktion \glqq insert(calendarId, body=None, sendNotifications=None, maxAttendees=None, sendUpdates=None, supportsAttachments=None, conferenceDataVersion=None)\grqq{} als \glqq body\grqq{} übergeben. Das Einfügen von Terminen bereitet jedoch ein Problem, wenn bereits zuvor abgefragte Termine, wieder in den google-Kalender zurückgeschrieben werden, da die ID des Termins beim Einfügen ignoriert und neu vergeben wird, somit muss die Entscheidung, ob der Termin bereits existiert, lokal getroffen werden. Hierfür werden von Google einmalig alle Termine abgefragt und die IDs in einer Liste gespeichert. Bevor ein Termin in den Kalender geschrieben wird, muss nun geprüft werden, ob die ID bereits in der Liste vorhandener Termine existiert. Wenn dies der Fall ist, wird der Termin herausgefiltert. Hierbei entsteht nur das Problem, dass Updates der Termine, die beispielsweise über einen weiteren Kalenderdienst vorgenommen wurden, aufgrund der identischen ID, nicht in den Google-Kalender geschrieben werden. \\
Serientermine werden bereits dekomprimiert abgefragt, um die Serienregeln nicht mithilfe eines Tools konvertieren zu müssen, da diese für jeden Service sehr speziell sind und sich stark voneinander unterscheiden. Ein Schreiben von Serienterminen in den Google-Kalender kann mit den in Tab.\ref{tab:google-recurrence} aufgelisteten Parametern innerhalb einer Liste erfolgen \cite{GoogleLLC.13.05.2019}. So ergibt beispielsweise [\grqq{}EXDATE;VALUE=DATE:20150610\grqq{},
\grqq{}RDATE;VALUE=DATE:20150609,20150611\grqq{},
\grqq{}RRULE:FREQ=DAILY;UNTIL=20150628;INTERVAL=3\grqq{}] eine Wiederholung des Termins vom angebenen Startdatum bis zum 28. Juni 2015 im 3-Tages-Rhythmus, wobei der Termin am 10. Juni ausfällt, jedoch am 9. und 11. Juni zusätzlich stattfindet. Anstelle dessen, könne auch hier Serientermine in Form einzelner Termine interpretiert werden, hierbei geht jedoch die Information der Zusammengehörigkeit der Termine verloren.

\begin{table}[h]
	\caption{Beispielhafter Aufbau der Serientermin-Regeln der Google-API}
	\label{tab:google-recurrence}
	\begin{tabularx}{\textwidth}{ | l | X | }
		\hline
		Parameter & Funktion \\ \hline
		EXDATE & Ausgeschlossene Tage \\ 
		RDATE & Zusätzliche Tage \\ 
		RRULE & Regel zur Angabe, wie häufig der Serien-Termin stattfindet \\
		\hline
	\end{tabularx}
\end{table}