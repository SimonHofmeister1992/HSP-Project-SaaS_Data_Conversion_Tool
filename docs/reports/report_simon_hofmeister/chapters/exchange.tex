\section{Microsoft-Exchange Kalender-Service}
Der Kalender-Dienst von Microsoft ist als Teil des Microsoft-Office Paketes durch den Client Outlook und als Server mit dem Namen Exchange sehr bekannt und sowohl im privaten Bereich als auch in Firmen weit verbreitet. Ein Vorteil des Kalenders ist die Spezialisierung auf die Terminplanung mit der Möglichkeit, Termine als Online-Konferenzen auszuweisen.

\subsection{Login in den Exchange-Service}
Da Microsoft Exchange auf vielen verschiedenen Plattformen installiert werden kann, wird hierbei neben den Login-Daten \glqq Domäne\textbackslash Benutzername\grqq{} und \glqq Passwort\grqq{} auch die Angabe benötigt, unter welcher URL der Exchange-Server erreichbar ist, ein Beispiel hierfür ist: \glqq exchange.othr.de\grqq{}. Daneben ist noch die Angabe der Email-Addresse notwendig, um die Verbindung zum konkreten Account herstellen zu können.
Diese Informationen reichen aus, um alle Daten eines beliebigen Exchange-Kalenders abzufragen und neue Informationen zu ergänzen. 

\subsection{Lesende und schreibende Abfragen der Exchange-API}
Mithilfe der Login-Informationen kann der konkrete Login in einen Account konfiguriert werden, womit die Rechte Termine einzusehen und neue Termine zu erstellen gewährt wird. Aus diesen Daten wird hierbei ein konkretes Account-Objekt erzeugt, mit dessen Hilfe man beispielsweise über die Kalendereinträge, aber auch Emails lokal iterieren kann. \\
Für die Einsicht in die Kalenderdaten gibt es zwei verschiedene Möglichkeiten, einerseits die Methode all(), die alle Termine in ihrer Grundform liefert, Serientermine werden hierbei nicht expandiert. Andererseits kann die Methode view() genutzt werden, diese expandiert Serientermine in die Einzeltermine, womit das zugrundeliegende Pattern nicht manuell interpretiert werden muss. Die hierfür genutzte exchangelib-Bibliothek bietet zudem noch vordefinierte Methoden, um den Datensatz nachzubearbeiten, beispielsweise kann die Reihenfolge an die Erstellreihenfolge der Termine angepasst werden, oder an eine alphabetische Prozessierung über den Titelnamen. Im Gegensatz zum Google-Kalender werden bei der Abfrage des Kalenders alle Subkalender berücksichtigt, könnten jedoch auch weiter über die Namen der Kalender eingeschränkt werden.\\
Das Schreiben von Terminen in einen Kalender gestaltet sich bei Microsoft Exchange Kalendern jedoch als deutlich schwieriger, da auf sehr viele Attribute ein rein lesender Zugriff gestattet ist. So können beispielsweise keine Informationen zum Änderungsdatum, IDs, oder Serienterminen angegeben werden. Dies führt dazu, dass Serientermine nur in Form von Einzelterminen neu eingefügt werden können. Eine Zusammengehörigkeit der Termine lässt sich im Kalender somit nicht mehr erkennen und die Termine lassen sich nicht mehr gemeinsam bearbeiten. Ob ein Eintrag bereits im Exchange-Service vorhanden ist, lässt sich hier jedoch nicht über die IDs erkennen, da diese in einer Abfrage nicht mitgegeben werden. Hierfür wird eine Liste erstellt, die die Sha-256-Werte aus dem Titel des Termins und dem Startdatum enthält. Hiermit sind Vergleiche deutlich performanter, als wenn für jeden existierenden Kalendereintrag mehrere Attribute verglichen werden müssten, zudem ist es sehr unwahrscheinlich, dass zwei Termine auf den selben Wert abgebildet werden. Um einen neuen Kalendereintrag anzulegen, wird ein Calendar-Item-Objekt angelegt, in welchem alle benötigten Attribute angegeben werden. Hierbei können beliebig viele Attribute im Konstruktor angegeben werden. Komplexere Datenstrukturen wie die Teilnehmer und teilweise nicht vorhandene Attribute, wie URLs zu Online-Konferenzlösungen, können auch außerhalb des Konstruktors zugewiesen werden. Um Attribute mit möglichen Standardwerten innerhalb des Konstruktors zuzuweisen, empfiehlt sich hierbei die Verwendung der if-else-expression in python, anstatt des if-else statements, welches innerhalb des Konstruktors nicht verwendet werden kann.
