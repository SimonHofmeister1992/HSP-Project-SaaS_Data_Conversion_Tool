\section{Konfigurationsmanagement}
Neben der Projektorganisation erleichtert auch die einheitliche Definition von Ordnerstrukturen und Variablennamen die Teamarbeit deutlich, da den Teammitgliedern die Einarbeitung in fremden Code durch den einheitlichen Standard einfacher fällt \cite{Versteegen.2013}. Zudem sind ein Versionskontrollsystem und Tests sinnvoll. 

\subsection{Ordnerstruktur}
Auf der obersten Ebene des Projekts\footnote{Das gesamte Projektverzeichnis kann unter dem nachfolgenden Hyperlink eingesehen werden: \url{https://drive.google.com/drive/folders/1Xtf8WVP2vQgZc7mYucWB7Ftz0_30_tmy?usp=sharing}} befinden sich die Ordner für die Dokumentation und den Source-Code, in letzterem befinden sich zum einen die Installationsskripte für die Services (nicht für die Installation von Python, Qt und MongoDB), der Ordner \glqq general \grqq, in dem sich die GUI, die Klassen für die Datenbankverbindungen, die abstrakten Klassen der höchsten Abstraktionsebene, baseApiInterface und dataObject, sowie weitere Tools befinden. Hier befinden sich auch die Ordner für die einzelnen Kategorien, die wiederum Ordner für die konkreten Services enthalten. 

\subsection{Versionskontrolle}
Um gemeinsam an dem Projekt arbeiten zu können, wurde das Versionskontrollsystem \glqq git \grqq ausgewählt, über welches die Projektstände auch synchronisiert wurden. Zudem ermöglicht git, dass bei lokalem Datenverlust der remote-Datenstand neu bezogen werden kann oder bei neu auftretenden Fehlern ein Vergleich mehrerer Versionen möglich ist. 

\subsection{Konventionen für Variablen}
Bei der Benennung der Klassen und Methoden fiel die Wahl auf die CamelCase-Schreibweise. Zudem wurden die Attribute für die Kalender-Datenobjekte mit Präfixe aus dem Hungarian-Apps-Style versehen, da Python keine strikten Datentypen definiert und es dem Programmierer die Arbeit erleichtert, wenn man anhand der Attributsnamen erkennt, welche Werte in einem Attribut erlaubt oder vorzufinden sind.\\
Mit circa 120 Attributen besitzen Einträge im Google-Kalender den größten Umfang an Möglichkeiten, Events in Microsoft Exchange haben noch ungefähr 80 Attribute während die Teamup-API nur 25 Attribute liefert. Hieraus ergibt sich auch die Reihenfolge, für das Mapping der Attribute. Wenn eine Eigenschaft vorhanden ist, wird diese wenn möglich, auf die Struktur der Google-Kalender Einträge gemapped, ergibt sich hier keine Übereinstimmung, wird in den Attributen der Microsoft Exchange Einträge gesucht und erst zuletzt in der Teamup-Datenstruktur. \\
Desweiteren wurde im Team das zu nutzende Datumsformat diskutiert, da hierfür verschiedene Angaben gebräuchlich sind. In der engsten Auswahl befanden sich die Zeitangaben in der UTC-Variante \glqq YYYY-MM-DDTHH:MM:SS.mmmZ\grqq sowie in der ISO-8601-Formatierung \glqq YYYY-MM-DDTHH:MM:SS$\pm$HH:MM\grqq. Der Vorteil der ISO-8601-Formatierung besteht in der Erhaltung der originalen Zeitzonen, während die UTC-Formatierung für eine einheitliche Formatierung im neuen System sorgt, jedoch konnten alle untersuchten Services mit beiden Varianten umgehen, sodass dort die Uhrzeit an den Client angepasst wird. Da außer der Rotation der Zeitzone keine weiteren Unterschiede existieren, wurde die UTC-Formatierung für alle Datumsangaben festgelegt.

\subsection{Tests}
Die Klassen, die die API nutzen und somit die Services nutzen oder in diese schreiben, wurden im Laufe der Entwicklung mehrmals manuell auf Fehler getestet, ebenso bei der Ausführung des Codes innerhalb des Teams durch weitere Teammitglieder und durch die Nutzung der graphischen Benutzeroberfläche.\\
Auf automatisierte Tests wurde in diesem Projekt bisher verzichtet, da nur generell getestet werden kann, ob die Datenbank erreichbar ist, die GUI funktioniert, und die Services auf einfacher Ebene mit lesenden und schreibenden Abfragen zurecht kommen. Die eigenen ApiInterface-Klassen mit teils mehr als 100 Attributen auf korrekte Funktion zu testen kann als sinnvoll erachtet werden, hätte jedoch den Rahmen des Projektes gesprengt, zudem müssen die APIs selbst nicht getestet werden, da man hier von einer gründlichen Testreihe durch die Service- und Frameworkanbieter ausgehen kann. \\