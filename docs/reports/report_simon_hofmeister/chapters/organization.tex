\section{Projektorganisation}
Da das Projekt von mehreren Entwicklern parallel umgesetzt wird, werden vor der konkreten Umsetzung des Projektes die gemeinsamen Absprachen im Team erläutert. Diese Absprachen umfassen die Aufgabenaufteilung, Konventionen bei der Entwicklung und die genutzte Software. Zudem wird auch die verwendete Hardware beschrieben, auf der das Programm entwickelt wurde, da die Weiterentwicklung oder Ausführung des Tools auf einem Computer mit vergleichbaren Ressourcen gewährleistet ist.

\subsection{Aufgabenverteilung}
Um aussagekräftige Ergebnisse zu erhalten, ob das Tool zur Speicherung und Transformation von Daten aus SaaS-Diensten generell geeignet ist, müssen genügend Services ausgewählt werden. Die Möglichkeiten sind durch die Kapazität von zwei Entwicklern jedoch auch beschränkt. Somit ergab sich die Aufteilung, dass jeder Entwickler drei beliebige Services innerhalb je einer Kategorie, Kalender oder Notizen, bearbeitet. Zudem gab es noch die zusätzlichen Aufgaben, eine Datenbankverbindung einzurichten und die Serviceabfragen für eine erleichterte Bedienung in einer graphischen Benutzeroberfläche darzustellen. Die resultierende Aufgabenverteilung wird in Tab. \ref{tab:aufgabenverteilung} dargestellt.
\begin{table}[h]
	\caption{Aufgabenverteilung im Team}
	\label{tab:aufgabenverteilung}
		\begin{tabularx}{\textwidth}{ | l | X | }
		\hline
		Projektmitglied & Aufgaben \\ \hline
		Simon Hofmeister & Entwicklung der GUI \\ 
		& Integration des Google Kalender-Service\\ 
		& Integration des Microsoft Exchange Kalender-Service \\ 
		& Integration des Teamup Kalender-Service  \\ \hline
		Stephan Nunhofer & Anbindung der MonogDB Datenbank \\
		&  Integration des Google Keep Notiz-Service\\
		&  Integration des Microsoft OneNote Notiz-Service \\
		&  Integration des Notion Notiz-Service \\ \hline
	\end{tabularx}
\end{table}
Die Services wurden unter anderem nach dem Kriterium der Zugänglichkeit zu den benötigten Informationen mittels einer Programmierschnittstelle (API) sowie deren Beschreibung und der unterschiedlichen Zuständigkeitsbereiche innerhalb der Kategorien ausgewählt. Das resultierende Gesamtbild des Systems soll die vielseitige Verwendbarkeit möglichst gut abbilden, jedoch sollten auch die Entwicklungsressourcen nicht unnötig stark für die Suche nach einer geeigneten Schnittstelle gebunden werden. 

\subsection{Konventionen für Variablen}
Da mehrere verschiedenen Dienste abgefragt werden und das Mapping der Attribute auf ein gemeinsames Datenschema erfolgen soll, ergibt sich die Notwendigkeit, eine Rangordnung zwischen den Diensten aufzustellen, um die Daten nicht mehrfach unter verschiedenen Namen oder in unterschiedlichen Datenstrukturen zu speichern, da dies den Speicher unnötig belasten und eine Einarbeitung in das Projekt erschweren würde. Mit circa 120 Attributen besitzen Einträge im Google-Kalender den größten Umfang an Möglichkeiten, Events in Microsoft Exchange haben noch ungefähr 80 Attribute während die Teamup-API nur 25 Attribute liefert. Hieraus ergibt sich auch die Reihenfolge, für das Mapping der Attribute. Wenn eine Eigenschaft vorhanden ist, wird diese wenn möglich, auf die Struktur der Google-Kalender-Einträge abgebildet. Ergibt sich hierbei keine Übereinstimmung, wird in den Attributen der Microsoft Exchange Einträge gesucht und erst zuletzt in der Teamup-Datenstruktur. \\
Desweiteren wurde im Team das zu nutzende Datumsformat diskutiert, da hierfür verschiedene Angaben gebräuchlich sind. In der engsten Auswahl befanden sich die Zeitangaben in der UTC-Variante \glqq YYYY-MM-DDTHH:MM:SS.mmmZ\grqq{} sowie in der ISO-8601-Formatierung \glqq YYYY-MM-DDTHH:MM:SS$\pm$HH:MM\grqq. Der Vorteil der ISO-8601-Formatierung besteht in der Erhaltung der originalen Zeitzonen, während die UTC-Formatierung für eine einheitliche Formatierung im neuen System sorgt, jedoch konnten alle untersuchten Services mit beiden Varianten umgehen, sodass dort die Uhrzeit an den Client angepasst wird. Da außer der Rotation der Zeitzone keine weiteren Unterschiede existieren, wurde die UTC-Formatierung für alle Datumsangaben festgelegt.

%\subsection{Kommunikation}
%Neben der klaren Aufgabenverteilung ist auch eine stetige Kommunikation unter den Projektmitgliedern nötig. Dies betrifft neben dem fachlichen und technischen Austausch von Informationen vor allem den Entwurf und die Implementierung der gemeinsamen Schnittstellen, sowie die Behebung entdeckter Fehler die bei den regelmäßigen (hier: manuellen) Tests der Module auffallen. 
%Hierzu verwendeten wir für den persönlichen Austausch untereinander die Sprach- und Textkanäle von Discord, zum Austausch mit unserem Betreuer die Zoom-Software und für die wichtigsten Informationen, sowie organisatorisches, Emails.

\subsection{Hardware und Software}
Um einen besseren Eindruck über das Gesamtsystem zu erlangen, wird im Folgenden noch die verwendete Hardware und Software erläutert. Diese Informationen sind nötig, wenn das Projekt entweder weiterentwickelt, oder verwendet werden soll. \\
Die verwendete Hardware entspricht der persönlichen Ausstattung der Entwickler, im Falle des Autors, ist dies ein Desktop-Rechner mit einem i7-6770k-Prozessor mit 4x4.0GHz Taktfrequenz und einem 32GB-DDR4-Arbeitsspeicher. Das verwendete Betriebssystem ist Linux Ubuntu 18.04 LTS. Für die Programmierung wurde Python in der Version 3.8.5 genutzt, die GUI wurde mittels QTCreator in der QT Software in der Version 5 erstellt. Bei der Datenbank fiel die Wahl auf MongoDB in der OpenSource-Serverversion 4.2. 

