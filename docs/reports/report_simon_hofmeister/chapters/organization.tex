\section{Projektorganisation}
Einer der wichtigsten Faktoren für ein erfolgreiches Projekt, ist eine definierte Projektorganisation. Hierzu gehören unter anderem die Aufgabenverteilung im Team und auf welchem Weg kommuniziert wird. Ebenso sind für eine Verifikation des Projektes, um die Ergebnisse bewerten und nachbauen zu können, die wichtigsten Eckdaten zur Hardware und Software vonnöten.

\subsection{Aufgabenverteilung}
Um dem Projekt einen Sinn zu geben und aussagekräftige Ergebnisse zu erhalten, müssen genügend Services ausgewählt werden. Die Möglichkeiten sind durch die Kapazität von zwei Entwicklern jedoch auch beschränkt. Somit ergab sich die Aufteilung, dass jeder Entwickler drei beliebige Services innerhalb je einer Kategorie, Kalender oder Notizen, bearbeitet. Zudem gab es noch die zusätzlichen Aufgaben, eine Datenbankverbindung einzurichten und die Serviceabfragen für eine erleichterte Bedienung in einer graphischen Benutzeroberfläche darzustellen. Die resultierende Aufgabenverteilung wird in Tab. \ref{tab:aufgabenverteilung} dargestellt.
\begin{table}[h]
	\caption{Aufgabenverteilung im Team}
	\label{tab:aufgabenverteilung}
		\begin{tabularx}{\textwidth}{ | l | X | }
		\hline
		Projektmitglied & Aufgaben \\ \hline
		Simon Hofmeister & Entwicklung der GUI \\ 
		& Integration des Google Kalender-Service\\ 
		& Integration des Microsoft Exchange Kalender-Service \\ 
		& Integration des Teamup Kalender-Service  \\ \hline
		Stephan Nunhofer & Anbindung der MonogDB Datenbank \\
		&  Integration des Google Keep Notiz-Service\\
		&  Integration des Microsoft OneNote Notiz-Service \\
		&  Integration des Notion Notiz-Service \\ \hline
	\end{tabularx}
\end{table}
Die Services wurden unter anderem nach dem Kriterium der Zugänglichkeit zu den benötigten Informationen mittels einer Programmierschnittstelle (API) sowie deren Beschreibung und der unterschiedlichen Zuständigkeitsbereiche innerhalb der Kategorien ausgewählt. Das resultierende Gesamtbild des Systems soll die vielseitige Verwendbarkeit möglichst gut abbilden, jedoch sollten auch die Entwicklungsressourcen nicht unnötig stark für die Suche nach einer geeigneten Schnittstelle gebunden werden. 

\subsection{Kommunikation}
Neben der klaren Aufgabenverteilung ist auch eine stetige Kommunikation unter den Projektmitgliedern nötig. Dies betrifft neben dem fachlichen und technischen Austausch von Informationen vor allem den Entwurf und die Implementierung der gemeinsamen Schnittstellen, sowie die Behebung entdeckter Fehler die bei den regelmäßigen (hier: manuellen) Tests der Module auffallen. 
Hierzu verwendeten wir für den persönlichen Austausch untereinander die Sprach- und Textkanäle von Discord, zum Austausch mit unserem Betreuer die Zoom-Software und für die wichtigsten Informationen, sowie organisatorisches, Emails.

\subsection{Hardware und Software}
Um einen besseren Eindruck über das Gesamtsystem zu erlangen, wird im folgenden noch die verwendete Hardware und Software erläutert. Diese Informationen sind nötig, wenn das Projekt entweder weiterentwickelt, oder verwendet werden soll. \\
Die verwendete Hardware entspricht der persönlichen Ausstattung der Entwickler, im Falle des Autors, ist dies ein Desktop-Rechner mit einem i7-6770k-Prozessor mit 4x4.0GHz Taktfrequenz und einem 32GB-DDR4-Arbeitsspeicher. Das verwendete Betriebssystem ist Linux Ubuntu 18.04 LTS. Für die Programmierung wurde Python in der Version 3.8.5 genutzt, die GUI wurde mittels QTCreator in der QT Software in der Version 5 erstellt. Bei der Datenbank fiel die Wahl auf MongoDB in der OpenSource-Serverversion 4.2. 
