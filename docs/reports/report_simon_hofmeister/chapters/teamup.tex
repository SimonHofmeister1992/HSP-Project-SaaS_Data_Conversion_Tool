\section{Teamup Kalender-Service}
Der Teamup-Kalender ist in der Reihe der hier betrachteten Kalender, das kleinste und unbekannteste Tool, womit dieses die analysierten Kalender gut ergänzt. Der Teamup-Kalender ist spezialisiert auf die Planung von gemeinsamen Terminen, wie Kursen, mit der Möglichkeit, dass sich Teilnehmer bei Erlaubnis selbst zu den Terminen einladen können. 

\subsection{Authentifizierung für einen Teamup-Kalender}
Um auf einen Kalender von Teamup zugreifen zu können, wird nur ein API-Key benötigt, zusammen mit der geheimen Kalender-ID. Der API-Key lässt sich nach dem Einloggen mit einem Teamup-Account auf \url{https://teamup.com/api-keys/request} erstellen, indem man die Email-Addresse, das Unternehmen, und einen optionalen Kommentar angibt und auf \glqq Create API Key \glqq klickt. Der API-Key wird direkt angezeigt und sollte gespeichert werden, da er für jede Abfrage des Kalenders sowie jedes Erzeugen eines neuen Termins benötigt wird. Die geheime Kalender-ID befindet sich bei der Betrachtung eines Kalenders im Browser in der URL. So ist bei dem Kalender auf \url{https://teamup.com/kst496bmane3rty9b7} die geheime ID \glqq kst496bmane3rty9b7\grqq{}. Jeder Benutzer kann mehrere Kalender mit unterschiedlichen Kalender-IDs anlegen, jeder dieser Kalender gliedert sich nochmals in definierbare Unterkalender. Durch die Kenntnis der Kalender-ID ist es jedoch jedem Nutzer möglich, auf jeden ihm bekannten Kalender einer beliebigen anderen Person mittels der API vollen Zugriff zu erhalten.

\subsection{Lesender und schreibender Zugriff auf die Teamup-API}
Bei der Abfrage des Teamup-Kalenders wird die REST-API mittels einer GET-Abfrage \url{https://api.teamup.com/KALENDER-ID/events} zusammen mit einem Teamup-Token, also dem erzeugten API-Key, abgefragt. Hierbei sind Einschränkungen des Zeitraums möglich, indem beides, das Start- und Enddatum mit angegeben werden. Als Antwort erhält man ein JSON-Dokument, das eine Liste mit den gefundenen Kalendereinträge enthält. Eine einfache Fehlerquelle ist hierbei das abgefragte Datum in ISO-8601-Format, da die Zeitzone hierbei nur angegeben wird, wenn diese nicht der Zeitzone UTC entspricht. Hierdurch ergeben sich zwei verschiedene Zeitformate die getrennt voneinander interpretiert werden müssen. \\
Beim Schreiben in den Teamup-Kalender wird eine POST-Anfrage an die Teamup-API geschickt, der Aufbau ist identisch zur GET-Abfrage, nur dass diese um den einzufügenden Termin in Form eines JSON-Dokuments ergänzt wird. Zudem beschränkt die Teamup-API schreibbare Termine auf solche, deren Startzeitpunkt in der Zukunft liegt. Diese Information ist wichtig, um vergangene Termine, beispielsweise vor einem Backup, zu exkludieren. Um doppelte Einträge im Kalender nach einer Schreibaktion zu vermeiden, müssen gesicherte Einträge mit den bereits online vorhandenen Terminen verglichen werden. Da die API keinerlei Informationen über IDs mitteilt, muss auch für diesen Service eine Liste mit Kennzeichen zur Eindeutigkeit erstellt werden. Die Entscheidung lag hierbei ebenso auf einem Sha-256-Mapping aus dem Start-Termin und dem Titel des Eintrags, da es unwahrscheinlich ist, dass zwei unterschiedliche Termine gleich benannt sind und zum selben Termin stattfinden. Ergeben sich aus den Sha-256-Werten für bereits existierende Termine und einem einzufügenden Termin die selben Werte, so wird angenommen, dass dieser bereits im Kalender vorhanden ist, und somit wird dieser nicht mehr neu eingefügt.