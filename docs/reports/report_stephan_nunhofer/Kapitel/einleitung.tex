%%Einleitung in das Thema der Arbeit

\chapter{Aktuelle Bedeutung von Software-as-a-Service-Anwendungen}

Mit \textit{Software-as-a-Service}-Anwendungen (SaaS) werden Programme bezeichnet, welche dem Kunden als Dienstleistung angeboten, jedoch auf der IT-Infrastruktur des Dienstleisters betrieben werden. Der Kunde kann also auf den Dienst zugreifen, ohne eine eigenen ausreichende Umgebung für dessen Betrieb zu besitzen. Meist erfolgt dieser Zugriff über Web-Schnittstellen \cite{McNee2007}. Durch die für den Kunden einfache Nutzung steigt der Bedarf nach der derartigen Angeboten. Beliebte Verwendungen sind Notizbuch- oder Kalender-Programme, da diese meist zentralisiert werden und von verschiedenen Orten erreichbar sein sollen. Generell ist diese Art von Dienst für die meisten Formen der zentralisierten Datenverwaltung geeignet. So braucht man auch beispielsweise bei einem neuen Rechner für ein Unternehmen nicht ein Verwaltungswerkzeug installieren, sondern kann direkt über einen Browser auf die Arbeitsdateien zugreifen. Ebenso muss der Anwender keine eigene Infrastruktur zum Betrieb des Dienstes bezahlen und versorgen, wodurch er auch Geld einsparen kann. Ein Nachteil von SaaS-Anwendungen ist hingegen die Abhängigkeit von der Verfügbarkeit durch den Anbieter. Jedoch können entsprechende Maßnahmen ergriffen werden, um längeren Ausfällen vorzubeugen. Dazu zählt ein redundantes System für den Dienst und eine stabile Internetanbindung des Kunden.

Das Beratungsunternehmen Gartner prognostiziert, dass Produkte auf Basis von SaaS auch in den folgenden zwei Jahren deutlich mehr Umsatz generieren werden \cite{Gartner2019}. SaaS-Anwendungen sind zudem der stärkste Umsatzzweig nach Gartner und konnten den Abstand zu den \textit{Infrastructure-as-a-Service}-Anwendungen (IaaS) ausbauen.

Da allerdings eine Vielzahl an Diensten jeweils für spezielle Zwecke vorhanden ist, sind die Daten oft zwischen diesen verteilt. Möchte nun ein Kunde das Angebot wechseln und stellt der neue Anbieter keine Konvertierungsmöglichkeit für die Daten des anderen Anbieters bereit, so muss der Kunde entweder die Daten selbstständig übertragen oder auf diese verzichten. Dieses Problem kann jedoch durch Extrahierung und Injektion der Informationen durch die gegebenen Anwendungsschnittstellen und eine externe Konvertierung in einem neuen Anwendungsprogramm behoben werden. Dabei zeigt sich schnell, dass die größten Probleme die Diversität der Datenstrukturen und die Einschränkungen der Schnittstellen durch die Anbieter ist. Die Daten mögen vorhanden sein, jedoch gibt es oft schon bei der Angabe der letzten Änderungszeit in den Metadaten verschiedene Vorgehensweisen. Es muss also für jeden Dienst eine Konvertierung der Informationen auf eine gemeinsames Datenformat stattfinden. Falls die Anwendung eine Exportfunktion hat, kann man dabei diese für einen Konvertierung in ein einfaches Speicherformat nutzten. \textit{Evernote} bietet beispielsweise einen Export als \textit{.enex}-Dateien (\textit{.xml}-Format) an. Dienste wie \textit{OneNote} von \textit{Microsoft} schränken zusätzlich die Migrationsmöglichkeiten durch direkte Eingabe in die Schnittstelle ein. Beispielsweise können keine Metadaten beschrieben werden. Gleichzeitig bietet diese Anwendung aber auch verschiedene Import-Programme zur Datenmigration von den größten anderen Anbietern an. Beide Probleme müssen dabei für jeden Dienst einzeln beachtet und behandelt werden. Das in dieser Arbeit beschriebene Projekt mit Simon Hofmeister und Stephan Nunhofer unter der Aufsicht von Prof. Dr. Johannes Schildgen hat eine prototypische Zusammenführung und Vereinheitlichung der Konvertierungsvorgänge der einzelnen Anwendungen zum Ziel. Dadurch sind beispielsweise auch Datentransfers zwischen Notiz- und Kalender-Diensten möglich. Wichtigste Eigenschaft soll dabei die einfache Erweiterbarkeit sein, um einfach neue Dienste zu diesem Werkzeug hinzufügen zu können. 