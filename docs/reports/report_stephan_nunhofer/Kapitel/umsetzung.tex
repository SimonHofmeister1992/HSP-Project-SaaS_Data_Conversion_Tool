\chapter{Umsetzung des Datentransformations- und Verteilungssystems}

Die Hauptaufgaben dieses Werkzeuges ist die Gewinnung der Daten aus den verschiedenen Diensten. Diese Daten sollen dann in einer Datenbank gespeichert werden und bei einer Abfrage zur Übertragung in einen anderen Dienst verändert werden können. Zur Ermöglichung dieses Ziels müssen die Unterschiedlichen Informationen aus den Anwendungen auf eine gemeinsame Datenmenge zusammengeführt werden. Darauf folgt eine Definition des Speichervorganges in die Datenbank und die Umsetzung der Rückführung in den selben oder einen anderen Dienst. Das Projekt fokussiert sich auf Dienstleister in den Bereichen Kalender- und Notizverwaltung, wobei zu jedem Bereich jeweils drei Angebote eingebunden werden. Die Notizverwaltung, auf welche sich diese Arbeit hauptsächlich konzentriert, wird dabei von \textit{Keep} (Google), \textit{OneNote} (Microsoft) und \textit{Notion} (Notion) vertreten.

\section{Zusammenführung der Daten}

Als erster Schritt werden die Daten aus den verschiedenen Schnittstellen aufgelistet. 