\chapter{Mögliche Anwendungen des Datentransformations- und Verteilungssystems}

Der Prototyp an sich kann schon eine einfache Variante der gewünschten Funktionalität betreiben. Jedoch kann dieser noch weiter verbessert und ausgebaut werden, um eine angenehmere und umfangreichere Bedienung zu ermöglichen. Dabei gibt es einmal noch Verbesserungen für den jetzigen Funktionsstand und zum anderen neue Möglichkeiten, welche das System gut ergänzen oder erweitern. 

\section{Mögliche Erweiterungen und Verbesserungen im jetzigen System}

Eine gute Erweiterung des jetzigen Programms wäre die Beachtung der Attributwerte. So kann man die Wertebereiche überprüfen, um fälschliche Angaben in der Benutzeroberfläche herauszufiltern. Ebenso könnten aber auch die Filter detailliertere Argumente übernehmen. Es wäre beispielsweise möglich, das Datum nicht als Zeichenkette sondern als tatsächliche Zeitpunkte zu vergleichen. Kennt der Nutzer nicht den Wertebereich eines Feldes und möchte einen eigenen Wert für ein Attribut in der grafischen Oberfläche setzen, ist eine Auswahlmöglichkeit aller Eingabemöglichkeiten von Vorteil. Diese ist nur bei kleinen Wertebereichen sinnvoll und kann über eine Abfrage des Datentyps im Dienst gewährleistet werden. Ein Beispiel hierfür ist eine Liste aller Farben für das \textit{color}-Attribut des \textit{Keep}-Notizdienstes, welches eine begrenzte Anzahl an Farben unterstützt. Die verschiedene Farben wären dann über ein \textit{Drop-Down} Menü auswählbar für diese Attribut.\\
Eine weiter Verbesserung der Nutzbarkeit stellt eine Veränderungsmöglichkeit der verwendeten \textit{pipeline} in der graphischen Oberfläche dar. Dabei wird die aus den Eingaben generierte Parameterkette dem Nutzer angezeigt und dieser kann diese nach Wunsch noch weiter verändern. Dies erfordert zwar Kenntnis der Datenbankdienstes, erlaubt erfahrenerer Nutzern jedoch eine deutliche Erweiterung der Anwendungsmöglichenkeiten des Systems. Die Eingabe des Nutzers muss dabei jedoch noch auf schädlichen Text oder unabsichtliche Löschvorgänge per Syntaxprüfung überprüft werden.\\
Soll es dem Nutzer nicht gestattet sein die \textit{pipeline} direkt zu bearbeiten, aber ihm trotzdem deren gesamte Funktionalität zur Verfügung gestellt werden, kann man eine geheimes Eingabefeld für die \textit{addAggOptions} aus Tabelle \ref{tab:params} einbauen. Dieses beinhaltet alle zusätzlichen Parameter der \textit{pipeline} und ermöglicht somit Zugriff auf alle möglichen Funktionalitäten der Datenbank. Normalerweise werden diese Optionen nur durch das System erstellt. Dem Kunden, welcher sich zusätzliche Funktionen wünscht, kann man so einfach eine Erweiterung anbieten, auch wenn er sich hierfür Wissen über die Funktionalität der \textit{pipeline} aneigenen muss. Nützliche Links zu der Dokumentation der jeweiligen Datenbank können jedoch auch zur Verfügung gestellt werden. 

\section{Mögliche weiterführende Anwendungen und Erweiterungen des Systems}
Das Programm setzt stark auf Flexibilität und Erweiterbarkeit. So kann man durch ein Hinzufügen von einer entsprechenden Dienst-Klasse und Speicherklasse einen neuen Dienst hinzufügen und ebenso über eine Datenbank-Klasse eine neue Datenbank einbinden. Theoretisch kann dieses Werkzeug also eine Schnittstelle für jede SaaS-Anwendung werden und den Datenaustausch zwischen ihnen gewährleisten. Wenn man nun den Transfer zu der Datenbank nicht über eine grafische Oberfläche, sondern über ein Event oder zeitgesteuert anstößt, kann die Datenbank auch als Backup für die Dienste fungieren. Die Informationen wären dadurch immer mindestens auf dem Stand der letzten automatisierten Ausführung. Wenn man nun zu bestimmten Zeiten Backups der Datenbank erstellt, kann man das System zu einem Versionsverwaltungssystem umbauen. Damit kann man beispielsweise auf alte Stände der Notizen zurückgreifen, falls die Einträge durch Unachtsamkeit gelöscht werden.
Möglich ist auch eine Analyse der gesammelten Einträge. Man kann dabei auch die Häufigkeiten der Aktualisierungen der einzelnen Dienste betrachten und so die aktivsten Programme herausfinden. Oder man betrachtet das Verhältnis der Einträge pro Typen des Eintrages. Gibt es also mehr Notizen als Termine, kann man betrachten, in welchen Zeiten der Nutzer besonders aktiv war. Durch diese Analysen kann man dann die besten Dienste finden. \\
Ein Nachteil des Systems ist jedoch die teilweise sehr eingeschränkte Rückführungsmöglichkeiten der Daten in die Anwendungen. Dadurch können eventuell wertvolle Metadaten in einigen Diensten nicht eingebunden werden. Dieses Problem kann dabei durch einen eigenen Dienst für das Werkzeug behoben werden, da dieser die Daten in ihrer internen Konvertierung darstellen kann. Er ist dann auch problemlos in der Lage, alle zusätzlichen Informationen aus den Diensten darzustellen, da er nicht an ein spezielles Muster oder andere Vorgaben gebunden ist. Eine solche Anwendung könnte dann auf alle Daten der eingebunden Dienste zugreifen und diese manipulieren und auslesen. So ist beispielweise eine Export-Möglichkeit der Datenbank im \textit{XML-} oder \textit{CSV}-Format denkbar. Auch können in diesen neue Funktionen wie das Analyseprogramm eingebunden werden. Man könnte nach dem Vorbild \textit{Maven} ein \textit{PlugIn}-System einbauen, womit neue Funktionalitäten einfach hinzugefügt werden können. Diese wurden dann in einen speziellen Ordner innerhalb des Programmverzeichnisses geladen und beispielsweise täglich ausgeführt. Eventuell stellt man dann auch, wie bei \textit{Maven}, durch eine Verbindungen mit einem Versionsverwaltungssystem eine öffentliche Bibliothek für diese Teilprogramme bereit.\\