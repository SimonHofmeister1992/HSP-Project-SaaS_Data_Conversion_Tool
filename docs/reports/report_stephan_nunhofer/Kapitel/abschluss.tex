\chapter{Mögliche zukünftige Anwendungen des Datentransformations- und Verteilungssystems}

Das Programm setzt stark auf Flexibilität und Erweiterbarkeit. So kann man durch ein Hinzufügen von einer entsprechenden Dienst-Klasse und Speicherklasse einen neuen Dienst hinzufügen und ebenso über eine Datenbank-Klasse eine neue Datenbank einbinden. Theoretisch kann dieses Werkzeug also eine Schnittstelle für jede SaaS-Anwendung werden und den Datenaustausch zwischen ihnen gewährleisten. Wenn man nun den Transfer zu der Datenbank nicht über eine grafische Oberfläche, sondern über ein Event oder zeitgesteuert anstößt, kann die Datenbank auch als Backup für die Dienste fungieren. Die Informationen wären dadurch immer auf einen konfigurierbar neuen Stand. Wenn man nun zu bestimmten Zeiten Backups der Datenbank erstellt, kann man das System zu einem Versionsverwaltungssystem umbauen. Damit kann man beispielsweise auf alte Stände der Notizen zurückgreifen, falls die Einträge durch Unachtsamkeit gelöscht werden. 
Möglich ist auch eine Analyse der gesammelten Einträge. Man kann dabei auch die Häufigkeiten der Aktualisierungen der einzelnen Dienste betrachten und so die aktivsten Programme herausfinden. Oder man betrachtet das Verhältnis der Einträge pro Typen des Eintrages. Gibt es also mehr Notizen als Kalender, kann man betrachten, in welchen Zeiten der Nutzer besonders aktiv war. Durch diese Analysen kann man dann die besten Dienste finden. \\
Nachteil des Systems sind jedoch die teilweise sehr eingeschränkte Rückführungsmöglichkeiten der Daten in die Anwendungen. Dadurch können eventuell wertvolle Metadaten in einigen Diensten nicht eingebunden werden. Dieses Problem kann dabei durch einen eigenen Dienst für das Werkzeug behoben werden, da dieser die Daten in ihrer internen Konvertierung darstellen kann. Er ist dann auch problemlos in der Lage, alle zusätzlichen Informationen aus den Diensten darstellen, da er nicht an ein spezielles Muster oder andere Vorgaben gebunden ist. Eine solche Anwendung könnte dann auf alle Daten der eingebunden Dienste zugreifen und diese manipulieren und auslesen. Auch können in diesen neue Funktionen wie das Analyseprogramm eingebunden werden. Man könnte nach dem Vorbild \textit{Maven} ein \textit{PlugIn}-System einbauen, womit neue Funktionalitäten einfach hinzugefügt werden können. Diese wurden dann in einen speziellen Ordner innerhalb des Programmverzeichnisses geladen und beispielsweise täglich ausgeführt. Eventuell stellt man dann auch, wie bei \textit{Maven}, durch eine Verbindungen mit einem Versionsverwaltungssystem eine öffentliche Bibliothek für diese Teilprogramme bereit.\\